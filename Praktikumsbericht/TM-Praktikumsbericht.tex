\documentclass[a4paper]{article}

\usepackage[utf8]{inputenc}
\usepackage[ngerman]{babel}
\usepackage[autostyle=true,german=quotes]{csquotes}
\usepackage{amsmath}
\usepackage{amsfonts}
\usepackage{amssymb}
\usepackage{graphicx}
\usepackage{longtable}
\usepackage{rotating}
\usepackage{tabularx}
\usepackage[left=2.5cm,right=2.5cm,top=2.5cm,bottom=2.5cm]{geometry}
\usepackage[colorlinks, linkcolor = black, citecolor = black, filecolor = black, urlcolor = blue]{hyperref} 
\author{Sebastian Gottwald, Simon Bordewisch}
\title{Text-Mining Praktikumsbericht}
\date{4. Februar 2016}
\begin{document}
\begin{titlepage}
\maketitle
\end{titlepage}

\section{Einleitung}
	Im Rahmen des Moduls Text-Mining hatten wir die Aufgabe einen Klassifikator für den Stanford Named Entity Recognizer (Stanford NER) zu erstellen. 
	Als Datengrundlage stand uns ein von Studenten erstelltes Programm zur Verfügung, welches eine kategorisierte Liste von Titel der deutschen Wikipedia erzeugt.
	Aufgabe war es, diese Daten aufzubereiten und anhand von Wikipediaartikeln aus einem XML-Dump Trainingsdaten für den Stanford NER Klassifikator zu erstellen, welcher Personen-, Organisations- und Ortsnamen erkennt.
	\\\\
	Der Stanford NER ist eine Java-Implementation eines Named Entity Recognizers (NER). 
	Ein NER markiert Wort-Sequenzenn in einem Text welche bestimmte Kategorien repräsentieren (z.B. Personen, Orte, Organisationen oder auch Gene und Proteine).
\section{Methodik und Vorgehen}
	\subsection{Auswahl der Vergleichsdaten}
		Insgesamt standen fünf verschiedene Programme des Vorjahrespraktika zur Verfügung, welche wir als Ausgangsdaten zur Erstellung der Klassifikators nutzen sollten.
		Die Entscheidung fiel auf das Programm, welches auf den Titeln der Wikipediaartikel arbeitet, da dieses eine umfangreiche und nahezu fehlerfreie Liste von Organisationen, sowie Personen- und Ortsnamen liefert.  
		Die anderen Programme waren schlecht oder gar nicht dokumentiert beziehungsweise ohne weiteres Zutun nicht lauffähig. 
		Das Programm ''Wiki\_ORG'' arbeitet zum Beispiel auf einer Datenbank welche nicht mehr vorhanden ist und im Programm ''WIKI\_ORT'' fehlen die zusätzlichen Abhängigkeiten.
		
	\subsection{Extraktion der Wikipedia Artikel}
		Als Datengrundlage zur Erstellung der Trainingsdaten des Klassifikators, dient der aktuelle Wikipedia Dump ''dewiki-latest-pages-articles.xml.bz2''. 
		Zur Extraktion der Daten wurde die StAX-API verwendet, da diese sich gut für große XML-Datenmengen eignet. 
		Beim parsen des Wikipediadumps wurden die Normdaten der Artikel überprüft, da Personenartikel die Bezeichnung ''Typ=p'', Ortsartikel die Bezeichung ''Typ=g'' und Körperschaftsartikel die Bezeichnung ''Typ=k'' in ihren Normdaten besitzen und somit die Artikel extrahieren werden können, welche wahrscheinlich die besten Ergebnisse beim finden von Wörtern liefert. 
		Zur Bereinigung der extrahierten Artikel wurde das Python-Script ''WikiExtraktor.py''\footnote{Quelle: \url{https://github.com/attardi/wikiextractor}} verwendet, welches den Klartext der Wikipediaartikel im Ordner ''Ergebnisse/AA/wiki\_00'' abspeichert.
		
	\subsection{Volltextsuche und Tagging}
		Um die Trainingsdaten erstellen zu können, musste zunächst ein Wörterbuch erstellt werden und der extrahierte Klartext getaggt werden.
		Ziel war es den Klartext so vorzubereiten, dass dieser anschließend in die Form gebracht werden konnte, die für den Standford NER benötigt wird.

		\subsubsection{Erstellung des Wörterbuchs}
			Um ein möglichst gutes Wörterbuch erstellen zu können, mit dem der Tagger arbeiten kann, haben wir uns dazu entschieden die Vergleichdaten der Titel-Gruppe aufzuarbeiten.
			Die Vergleichsdaten, die alle Körperschaften, Orte und Personen in einer Datei enthielten, wurden von einem Parser\footnote{Vergleich package ''parsetitlenorm''} zunächst in die verschiedenen Kategorien aufgeteilt, bereinigt und anschließend in drei verschiedenen Dateien (eine pro Kategorie) im CSV-Format gespeichert. 
	 ------ Bei der Bereinigung wurden Wörter, Einträge, die in einer Blacklist stehen, rausgefiltert. ------
			Ursprünglich sollten hier die Namen der Personen nach Vor- und Nachname getrennt werden, dies wurde allerdings im späteren Entwicklungsverlauf beim Einlese-Prozess der CSV-Dateien in den Tagger realisiert.

		\subsubsection{Einlesen des Klartextes und Tagging}
			Das package ''mapping'' dient dazu die Klartexte, die zu Trainingszwecken genutzt werden sollen, einzulesen und die Wörter einzeln zu untersuchen und die Wörter zu markieren, die als ein ''Match'', also eine Übereinstimmung mit dem Wörterbuch, erkannt wurden.
			Die Definition des Matches unterlief dabei mehrern Iterationen. 
			Zunächst wurden nur vollständige Übereinstimmungen als Match angesehen.
			Dies lieferte jedoch zu schlechte Ergebnisse, da zum Beispiel gebeugte Namen (z.B. ''Merkels'') nicht erkannt werden.
			Daraufhin wurde Levenshtein-Distanz zur Berechnung der Ähnlichkeit benutzt, um die obigen Ergebnisse zu verbessern. 
			Später wurde das Programm so abgeändert, dass die Ähnlichkeit der Suffixe der Wörter (die letzten drei Buchstaben der Wörter) berechnet wurde, während der Rest des Wortes exakt übereinstimmen muss. 
			Für das Ähnlichkeitsmaß wurden die Ergebnisse der obig genannten Gruppen verwendet, welche auf der Levenshtein-Distanz relativ zur Gesamtlänge des Eintrages basiert, sodass die Ähnlichkeit ca 83\% betragen muss.
			Zusätzlich mit der Implementierung von Gewichtungen, die sowohl die Ähnlichkeit des Eintrags als auch das Fehlen von zum Beispiel der Vornamen (zum Beispeil im Fall des Eintrages ''Angela Merkel'', wenn im Text nur ''Merkel'' vorkommt) gewichtet, liefert das Matching gute Ergebnisse. 
			Diese Matchings wurden von dem Programm markiert. 
			Hierfür wurden XML-ähnliche Tags verwendet(vgl. Bild).
			Der daraus resultierende Text mit den Tags wird anschließend in eine Datei geschrieben (Standard: ''Ergebnisse/Mapped.out''). 
			Diese ermöglicht durch seine Struktur ein Debugging der Texte, da die XML-Tags den im Wörterbuch gefundenen Match enthalten.

	\subsection{Erstellung der Traningsdaten}
		Das im vorherigem Abschnitt erwähnte getaggte Dokument wurde zunächst in einzelne Wörter aufgeteilt. 
		Dafür wurde der von der Stanford Core NLP mitgelieferte Tokenizer benutzt. 
		Dieser transformiert den Text dahingehend, dass er pro Zeile ein Wort schreibt. 
		Wir fügten danach an jedes Wort einen Tabulator und den Buchstaben 'O', welcher als default-Buchstabe des Stanford NER genutzt wird, an.
		Im nächsten Schritt parsten wir den entstandenen tokenisierten Text und ersetzten den Buchstaben 'O' der Wörter zwischen den Tags, durch die jeweilige Kategorie. 
		Danach schrieben wir alle Zeilen die keine Tags enthalten, also die richtig formatierten Trainingsdaten, in die Datei ''TrainingsData.out''.
	
	\subsection{Erstellung des Klassifikators}
		Nachdem die Trainingsdaten maschinell erstellt wurden, wurde der Stanford NER trainiert.
		Im Hauptprogramm wird die Main-Methode des Stanford NER Klassifikators aufgerufen und eine Propertydatei (''Ressourcen/default.prop'') übergeben. 
		Diese Propertydatei gibt Eigenschaften des Trainings und des zu erstellenden Klassifikators an.

\section{Probleme bei der Lösung der Aufgabe}
	Beim extrahieren des Klartextes aus den gefilterten Artikeln mit dem Programm ''WikiExtraktor.py'' werden zum bei manchen Artikeln Teile des Textes verworfen. 
	Wir konnten leider die Ursache von diesem Fehler nicht finden. 
	Da wir jedoch auf einer sehr großen Menge an Daten arbeiten, entschieden wir uns diesen Bug zu ignorieren, da das Training auf einem Teilartikel wahrscheinlich keine Auswirkung auf unser Ergebnis hat.
	
	Zum Auffinden der zu markierenden Wörter in unserem Text benutzten wir zunächt ''exact-matches''. Vornamen und Nachnamen die einzeln im Text standen und die Beugung von Wörtern wurden dadurch leider nicht erkannt. 
	Um dies zu verbessern, spalteten wir die Liste der Personennamen in Vornamen und Nachnamen auf und benutzten die Levenstein Distanz um Beugungen zu erkennen. 
	Hierbei ergab sich ein erneutes Problem die auf die Mehrdeutigkeit von Wörtern, speziell Nachnamen, zurückzuführen ist, denn nun wurden Wörter markiert welche eine andere Beudeutung haben. 
	Um dies zu verhindern wurde eine Art Bestrafung bei der Erkennung von Nachnamen eingeführt, wenn kein Vornamen vorhanden ist. 
	Generell ist die Mehrdeutigkeit von Wörtern ein großes Problem beim markieren und wir erlangen somit viele falsch positive Ergbnisse. 
	Desweiteren werden durch die Levenstein Distanz ebenfalls viele Wörter auch falsch markiert. 
	Als Beispiel dient das Wort ''Triumph'' welches sowohl eine Motorradmarke also Organisation, als auch Triumph im Sinne von Siegen ist. Das Wort ''Stadt'' wird als Ort markiert wird, da es einen Ort Namens ''Stade'' gibt und die Leventstein Distanz lediglich 1 beträgt.
	
	Ein weiteres Problem stellt die Laufzeit unseres Programmes dar. 
	Die zwei am längsten dauernden Funtionen sind das Extrahieren der Wikipediaartikel und das erstellen des Klassifikators mit der Stanford NER Bibliothek.
\section{Ergebnisse}
	Alle maschinell erstellten Klassifizierer wurden auf einem per Hand annotierten Goldstandart verglichen. 
	Dieser Goldstandart bestand aus 100 Beispielsätzen (1530 Wörter und Satzzeichen) aus Nachrichtenartikeln oder Beispielsätzen des Projektes ''Wortschatz'' der Universität Leipzig. 
	Desweiteren haben wir unsere Klassifikator gegen den auf der Stanford NER Webseite erhältlichen deutschen Klassifikator  verglichen. Die Ausführlichen Ergebnisse sind in der Tabelle \nameref{Tabelle Ergebnisse} zu finden.
	
	Man kann erkennen, dass der Klassifikator welcher Vor- und Nachnamen getrennt betrachtet und mit der Property-file des deutschen Klassifikators trainiert wurde, am besten Abschneidet. 
	Desweiteren sieht man einen Unterschied zwischen den Klassifikatoren welche Vor- und Nachnamen getrennt betrachten zu diesen, die Namen als ganzes betrachten. 
	Mit ersterem werden mehr Wörter richtig markiert, dies ist dadurch zu erklären, dass oftmals Vor und Nachnamen nur einzeln im Text audtauchen. 
	Auf der anderen Seite werden dadurch auch mehr falsch positive, also Wörter die keine Kategorie repräsentieren, getaggt. 
	Dies ist auf die Mehrdeutigkeit der Nachnamen von Personen zurückzuführen. 
	Nicht desto Trotz kommen wir zu dem Schluss, dass eine Trennung nach Vor und Nachnamen besser ist. 
	Man könnte die Ergebnisse vielleicht noch verbessern, indem man beim erstellen der Traningsdaten die linken und rechten Nachbarn der Wörter anschaut und somit auf Fehler schließen kann. 
	
	
	



\begin{sidewaystable}
\label{Tabelle Ergebnisse}
\caption{Ergebnistabelle}
\begin{longtable}{|c|c|c|c|c|c|c|c|c|}
\hline 
Anzahl Artikel & Variante & Property Datei & Precision & Recall & F1 & TP & FP & FN \\ 
\hline 
150 & Personen Vor- Nachname getrennt & german.dewac\_175m\_600 & 40,98\% & 47,19\% & 43,86\% & 84 & 121 & 94 \\ 
\hline 
150 & Personen Vollnamen & german.dewac\_175m\_600 & 50,94\% & 30,34\% & 38,03\% & 54 & 52 & 124 \\ 
\hline 
150 & Personen Vor- Nachname getrennt & default & 39,11\% & 44,38\% & 41,58\% &  79 & 123 & 99 \\ 
\hline 
150 & Personen Vollname & default & 49,53\% & 29,78\% & 37,19\% & 51 & 54 & 125 \\ 
\hline 
  &  & German-Classifier-DEWAC & 65,81\% & 57,30\% & 61,26\% & 102 & 53 & 76 \\ 
\hline 
  &  & German-Classifier-HGC & 61,22\% & 50,56\% & 55,38\% & 90 & 57 & 88 \\ 
\hline
\end{longtable} 
\end{sidewaystable}


\end{document}
