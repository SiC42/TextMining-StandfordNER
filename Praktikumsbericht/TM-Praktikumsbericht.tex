\documentclass[a4paper]{article}

 \usepackage[utf8]{inputenc}
  \usepackage[ngerman]{babel}
  \usepackage[autostyle=true,german=quotes]{csquotes}
\usepackage{amsmath}
\usepackage{amsfonts}
\usepackage{amssymb}
\usepackage{graphicx}
\usepackage[colorlinks, linkcolor = black, citecolor = black, filecolor = black, urlcolor = blue]{hyperref} 
\author{Sebastian Gottwald, Simon Bordewisch}
\title{Text-Mining Praktikumsbericht}
\date{4. Februar 2016}
\begin{document}
\maketitle
\newpage
\section{Einleitung}
Im Rahmen des Moduls Text-Mining hatten wir die Aufgabe einen Klassifikator für den Stanford Named Entity Recognizer (Stanford NER) zu erstellen. Als Datengrundlage standen uns ein, von Studenten erstelltes, Programm zur Verfügung, welches eine kategorisierte Liste von Titel der deutschen Wikipedia erzeugt. Unsere Aufgabe war es, diese Daten aufzubereiten und anhand von Wikipediaartikeln aus einem XML-Dump, Trainingsdaten für den Stanford NER Klassifikator zu erstellen, welcher Personen-, Organisations- und Ortsnamen erkennt
\\\\
Der Stanford NER ist eine Java-Implementation eines Named Entity Recognizers (NER). Ein NER markiert Wort-Sequenzenn in einem Text welche bestimmte Kategorien repräsentieren (z.B. Personen, Orte, Organisationen oder auch Gene und Proteine).
\section{Methodik und Vorgehen}
\subsection{Aufarbeitung der Vergleichsdaten}
Wir hatten insgesamt vier verschiedene Programme des Vorjahrespraktikas zur Verfügung. Wir entschieden uns das Programm zu nutzen, welches auf den Titeln der Wikipediaartikel arbeitet, da dieses am leichtesten zu bedienen war und wir eine umfangreiche und nahezu fehlerfreie Liste von Organisationen sowie Personen- und Ortsnamen erhielten. Wir mussten zunächst diese Liste aufarbeiten, sodass sie ein für uns nutzbares Format besitzt. Wir speicherten die einzelnen Entitäten in einer TreeMap zur weiterverarbeitung. Dadurch erziehlen wir gute Zeiten bei der Suche nach einem Wort.
\subsection{Extraktion der Wikipedia Artikel}
Als Datengrundlage zur erstellung der Trainingsdaten des Klassifikators, diente uns der aktuelle Wikipedia Dump ''dewiki-latest-pages-articles.xml.bz2''. Zum extrahieren der Daten nutzten wir StAX-API da diese sich gut für Große Datenmengen eignet. Beim parsen de Wikipediadumps haben wir die Normdaten der Artikel überprüft, da Personenartikel die Bezeichnung ''Typ=p'', Ortsartikel die Bezeichung ''Typ=g'' und Körperschaftsartikel die Bezeichnung ''Typ=k'' in ihren Normdaten besitzen und wir somit die Artikel extrahieren können, welche uns wahrscheinlich beste Ergebnisse beim finden von Wörtern liefern. Zur Bereinigung der extrahierten Artikel verwendeten wir das Python-Script ''WikiExtraktor.py'' (Quelle: \url{https://github.com/attardi/wikiextractor}), welches den Klartext der Wikipediaartikel im Ordner ''Ergebnisse/AA/wiki\_00'' abspeichert.

\subsection{Suche der Entities anhand von Vergleichsdaten}
\subsection{Erstellung der Traningsdaten}
Zur Erstellung der Trainingsdaten, teilten wir zunächst unseren markierten Text in einzelne Wörter auf. Dafür benutzten wir den von der Stanford Core NLP mitgelieferten Tokenizer. Dieser Transformiert den Text dahingehend, dass er in eine Zeile ein Wort schreibt. Wir fügten danach an jedes Wort einen Tabulator und den Buchstaben 'O' welcher als Default Buchstabe des Stanford NER genutzt wird.
Im nächsten Schritt parsten wir den entstandenen tokenisierten Text und ersetzten den Buchstaben 'O' der Wörter zwischen den Tags, durch die jeweilige Kategorie. In diesem Schritt löschten wir die tags aus dem Text und schrieben das Ergebnis, also die richtig formatierten Trainingsdaten, in die Datei ''TrainingsData.out''.

\subsection{Erstellung des Klassifikators}
In diesem Schritt trainierten wir den Stanford NER anhand unserer erstellten Trainingsdaten. Dafür rufen wir die Main Methode des Stanford NER Klassifikators auf und übergeben ihm eine propertyfile. Diese gibt Eigenschaften des Trainings und des zu erstellenden Klassifikators an.

\section{Probleme bei der Lösung der Aufgabe}
\section{Ergebnisse}
\end{document}